\documentclass[a4paper]{article}
\usepackage[english]{babel}
\usepackage{graphicx}
\usepackage{multicol}
\usepackage{amsmath}
\usepackage{hyperref}
\usepackage{amsthm}
\usepackage{geometry}
\geometry{a4paper} 
\usepackage{fancyhdr}
\usepackage{xcolor}
\usepackage{amssymb}
\usepackage{multicol}
\theoremstyle{definition}
\newtheorem{definition}{Defintion}[section]
\newtheorem{exmp}{Example}[section]
\newtheorem{theorem}{Theorem}

\begin{document}
\author{Fractals}
\title{\textbf{Functions}}
\maketitle
\tableofcontents
\noindent
\section{Introduction}
\subsection{Defintions}

\begin{definition}
    A function \(f\) from a set \(X\) to a set \(Y\) is a relation that assigns to each element in
    set \(X\) exactly one element in set \(Y\).
\end{definition}

\begin{definition}
    The domain is the set of \(X\) (a.k.a. the input).
\end{definition}

\begin{definition}
    The range is the subset of Y (a.k.a. the output).
\end{definition}

\end{document}
