\documentclass[a4paper]{article}
\usepackage[english]{babel}
\usepackage{graphicx}
\usepackage{multicol}
\usepackage{amsmath}
\usepackage{hyperref}
\usepackage{amsthm}
\usepackage{geometry}
\geometry{a4paper} 
\usepackage{fancyhdr}
\usepackage{xcolor}
\usepackage{amssymb}
\usepackage{multicol}
\theoremstyle{definition}
\newtheorem{definition}{Defintion}[section]
\newtheorem{exmp}{Example}[section]
\newtheorem{theorem}{Theorem}

\begin{document}
\author{Fractals}
\title{\textbf{Functions}}
\maketitle
\tableofcontents
\noindent
\section{Introduction}
\subsection{Defintions}

\begin{definition}
    A function \(f\) from a set \(X\) to a set \(Y\) is a relation that assigns to each element in
    set \(X\) exactly one element in set \(Y\).
\end{definition}

\begin{definition}
    The domain is the set of \(X\) (a.k.a. the input).
\end{definition}

\begin{definition}
    The range is a subset of Y (a.k.a. the output).
\end{definition}

\subsection{Existence of a Function}
\begin{theorem}[\textbf{Vertical Line Test}]
    if you can draw a Vertical line that passes through more than one point of a
    relation on a grap, it's not a function, if you cannot, it's a function.

\end{theorem}


\begin{exmp}
    what the domain and range of the function \(f(x) = \sqrt{16 - x^2}\)?
\end{exmp}

\(sloution\) Note that if \(a < 0 \), then \( \sqrt{a}\) is undefined for reals,
Thus, \(16 - x^2 \ge 0 \Rrightarrow \) \framebox[\width]{
    \(-4 \le x \le 4\)
} since \(x^2 \ge 0\), we have that \(0 \le 16 - x^2 \le 16\),
so the range is \framebox[\width]{ \(0 \le y \le 4\) }


\section{Combinations of Functions}
\begin{theorem}[\textbf{common function Combinations}]
    The following are some common combinations of functions:
    \begin{itemize}
        \item \textbf{Sum} \((f+g)(x) = f(x) + g(x)\)
        \item \textbf{Differnce} \((f-g)(x) = f(x) - g(x)\)
        \item \textbf{Product} \((fg)(x) = f(x)g(x)\)
        \item \textbf{Quotient} \((\dfrac{f}{g} )(x) = \dfrac{f(x)}{g(x)} \)
              where \(g(x) \ne 0\)
        \item \textbf{Compostion} \((f\circ g)(x) = f(g(x))\)
    \end{itemize}

\end{theorem}


\subsection{Domain and Range of a Composite Function}
\section{Types of Functions}
\subsection{Piecewise-Defined Function}
\section{Properties of Functions}
\subsection{Odd and Even Functions}
\subsection{Periodic Functions}
\section{Inverse Functions}
\subsection{Existence of an Inverse Function}
\end{document}
