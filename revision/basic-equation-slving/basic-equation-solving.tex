\documentclass[a4paper]{article}
\usepackage[english]{babel}
\usepackage{graphicx}
\usepackage{multicol}
\usepackage{amsmath}
\usepackage{hyperref}
\usepackage{amsthm}
\usepackage{geometry}
\geometry{a4paper} 
\usepackage{fancyhdr}
\usepackage{xcolor}
\usepackage{amssymb}
\usepackage{multicol}
\theoremstyle{definition}
\newtheorem{exmp}{Example}[section]
\newtheorem{theorem}{Theorem}

\begin{document}
\author{ \textbf{Fractals}}
\title{\textbf{Basic Equation Solving}}
\maketitle
\tableofcontents
\noindent
\section{Basic Equation Solving}
\subsection{Algebraic Manipulation}
To introduce the topic of \textbf{algebraic manipulation},
let us start with a little known example: \textbf{egyptian fractions}

\begin{theorem}[\textbf{Egyptian Fractions}]
    For all \(a, b \) where \(ab \ne 1 \)
    \[
        \dfrac{a}{ab - 1 } = \dfrac{1}{b( ab - 1)} + \dfrac{1}{b}.
    \]
    From here, we can see that putting things together (factoring) is just as important as taking them
    apart (distributing). Now, let us turn the power of products:
\end{theorem}
\begin{exmp}
    For positive real numbers \(a, b \),
    \[
        a + \dfrac{1}{b} = 4,
    \]
    \[
        b + \dfrac{1}{a} = 5,
    \]
    Find \(ab + \dfrac{1}{ab} \). \\

    \noindent
    \(Solution\). It is very easy to get lost in the problem if we directly try to solve for a and b. Instead,
    let us multiply the equations:
    \[
        (a+\dfrac{1}{b})(b+\dfrac{1}{a}) = 4(5) = 20.
    \]
    \[
        ab + \dfrac{a}{a} + \dfrac{b}{b} + \dfrac{1}{ab} = ab + \dfrac{1}{ab} + 2 = 20,
    \]
    \begin{center}
        \framebox[1\width]{ \( ab + \dfrac{1}{ab} = 18\). }
    \end{center}
\end{exmp}
\begin{theorem}
    Let \(x, y \) be nonzero real numbers such that \(x + y = a\) and \(xy = b\) Then,
    \[
        x^2 + y^2 = a^2 -2b,
    \]
    \[
        (x+1)(y+1) = a + b + 1,
    \]
    \[
        x^2 + xy^2 = ab,
    \]
    \[
        | x - y | = \sqrt{a^2 - 4b},
    \]
    \[
        x^3 + y^3 = a^3 - 3ab,
    \]
    \[
        \dfrac{1}{x} + \dfrac{1}{y} = \dfrac{a}{b},
    \]
\end{theorem}
\subsection{Quadratic Equations}
A polynomial is an equation of the following form:
\[
    a_n x^n + a_{n-1} x^{n-1} + \cdots + a_1 x + a_0 ,
\]
where
\[
    a_0, a_1, \dots, a_n
\]
are constants. A quadratic equation is a polynoimal with \(n = 2\):
\[
    ax^2 + bx + c = 0,
\]
A common way to solve a quadratic equation is to use the quadratic formula:
\begin{theorem}[\textbf{Quadratic Formula}]
    For the equation \(ax^2 + bx + c = 0\), the roots \(x_1, x_2 \) must be equal to
    \[
        x_1 = \dfrac{-b + \sqrt{b^2 - 4ac}}{2a},
    \]
    \[
        x_2 = \dfrac{-b - \sqrt{b^2 - 4ac}}{2a},
    \]
\end{theorem}

\end{document}