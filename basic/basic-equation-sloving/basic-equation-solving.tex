\documentclass[a4paper]{article}
\usepackage[english]{babel}
\usepackage{graphicx}
\usepackage{multicol}
\usepackage{amsmath}
\usepackage{hyperref}
\usepackage{amsthm}
\usepackage{geometry}
\geometry{a4paper} 
\usepackage{fancyhdr}
\usepackage{xcolor}
\usepackage{amssymb}
\usepackage{multicol}
\theoremstyle{definition}
\newtheorem{exmp}{Example}[section]
\newtheorem{theorem}{Theorem}
\newtheorem{problem}{Problem}

\begin{document}
\author{ \textbf{Fractals}}
\title{\textbf{Basic Equation Solving}}
\maketitle
\tableofcontents
\noindent
\section{Basic Equation Solving}
\subsection{Algebraic Manipulation}
To introduce the topic of \textbf{algebraic manipulation},
let us start with a little known example: \textbf{egyptian fractions}

\begin{theorem}[\textbf{Egyptian Fractions}]
    For all \(a, b \) where \(ab \ne 1 \)
    \[
        \dfrac{a}{ab - 1 } = \dfrac{1}{b( ab - 1)} + \dfrac{1}{b}.
    \]
    From here, we can see that putting things together (factoring) is just as important as taking them
    apart (distributing). Now, let us turn the power of products:
\end{theorem}
\begin{exmp}
    For positive real numbers \(a, b \),
    \[
        a + \dfrac{1}{b} = 4,
    \]
    \[
        b + \dfrac{1}{a} = 5,
    \]
    Find \(ab + \dfrac{1}{ab} \). \\

    \noindent
    \(Solution\). It is very easy to get lost in the problem if we directly try to solve for a and b. Instead,
    let us multiply the equations:
    \[
        (a+\dfrac{1}{b})(b+\dfrac{1}{a}) = 4(5) = 20.
    \]
    \[
        ab + \dfrac{a}{a} + \dfrac{b}{b} + \dfrac{1}{ab} = ab + \dfrac{1}{ab} + 2 = 20,
    \]
    \begin{center}
        \framebox[1\width]{ \( ab + \dfrac{1}{ab} = 18\). }
    \end{center}
\end{exmp}
\begin{theorem}
    Let \(x, y \) be nonzero real numbers such that \(x + y = a\) and \(xy = b\) Then,
    \[
        x^2 + y^2 = a^2 -2b,
    \]
    \[
        (x+1)(y+1) = a + b + 1,
    \]
    \[
        x^2 + xy^2 = ab,
    \]
    \[
        | x - y | = \sqrt{a^2 - 4b},
    \]
    \[
        x^3 + y^3 = a^3 - 3ab,
    \]
    \[
        \dfrac{1}{x} + \dfrac{1}{y} = \dfrac{a}{b},
    \]
\end{theorem}
\subsection{Quadratic Equations}
A polynomial is an equation of the following form:
\[
    a_n x^n + a_{n-1} x^{n-1} + \cdots + a_1 x + a_0 ,
\]
where
\[
    a_0, a_1, \dots, a_n
\]
are constants. A quadratic equation is a polynoimal with \(n = 2\):
\[
    ax^2 + bx + c = 0,
\]
A common way to solve a quadratic equation is to use the quadratic formula:
\begin{theorem}[\textbf{Quadratic Formula}]
    For the equation \(ax^2 + bx + c = 0\), the roots \(x_1, x_2 \) must be equal to
    \[
        x_1 = \dfrac{-b + \sqrt{b^2 - 4ac}}{2a},
    \]
    \[
        x_2 = \dfrac{-b - \sqrt{b^2 - 4ac}}{2a},
    \]
\end{theorem}
\begin{theorem}
    For the equation \(ax^2 + bx + c = 0\), we have the following cases:
    \begin{itemize}
        \item If \(b^2 - 4ac > 0\), we have \textbf{two real roots.}
        \item If \(b^2 - 4ac = 0\), we have \textbf{one real root.}
        \item If \(b^2 - 4ac < 0\), we have \textbf{no real roots.}
    \end{itemize}
    Using the Quadratic Formula, we can calculate the sum of roots and product of roots:
\end{theorem}
\begin{theorem}
    For the equation \(ax^2 + bx + c = 0\), the sum of roots is:
    \begin{displaymath}
        x_1 + x_2 = \dfrac{-b + \sqrt{b^2 - 4ac}}{2a} + \dfrac{-b + \sqrt{b^2 - 4ac}}{2a}
        = \dfrac{-2b}{2a} = - \dfrac{b}{a}
    \end{displaymath}
    \begin{displaymath}
        x_1x_2 = \dfrac{-b^2 + 4ac}{2a} \times \dfrac{-b^2 + 4ac}{2a} =
        \dfrac{-(-b)^2 - (b^2 - 4ac)}{4a^2} = \dfrac{c}{a}
    \end{displaymath}
\end{theorem}
\begin{theorem}
    For any polynoimal
    \[
        a_n x^n + a_{n-1} x^{n-1} + \cdots + a_1 x + a_0 ,
    \]

    \textbf{The sum of roots is}: \( - \dfrac{\text{seconed coeffeicent}}{\text{first coeffeicent}}
    = - \dfrac{a_{n-1}}{a_n}\)
    and \textbf{the product of roots is:} \(\dfrac{\text{last coeffeicent}}{\text{first coeffeicent}}
    = \dfrac{a_0}{a_n} \) \\

    \noindent
    For those who are looking for a more advanced and more powerful theorem, we can generalize
    this formula:
\end{theorem}

\begin{theorem}[\textbf{Vieta’s Formulas}]
    for any polynoimal
    \[
        a_n x^n + a_{n-1} x^{n-1} + \cdots + a_1 x + a_0 ,
    \]
    let \(r_1, r_2, \dots r_n\) (an \(n\)-degree equation has n different roots).
    Vieta’s formulas state that
    \[
        a_n = a_n
    \]
    \[
        a_{n-1} = -a_n(r_1 + r_2 + \cdots + r_n)
    \]
    \[
        a_{n-2} = a_n(r_1r_2+r_1r_3+\cdots + r_{n-1}r_n)
    \]
    \[
        \vdots
    \]
    \[
        a_0 = (-1)^n a_n(r_1r_2\cdots r_n)
    \]
\end{theorem}

\section{Problems}

\begin{problem}
Let \(x\) be a real number such that \(x+\dfrac{1}{x} = \sqrt{2020}\).
What is \(x^2 + \dfrac{1}{x^2}\)?
\end{problem}

\begin{problem}
Two non-zero real numbers, a and b, satisfy \(ab = a - b\).
Which of the following is a possible value of \(\dfrac{a}{b}
+ \dfrac{b}{a} -ab \)?
\end{problem}

\begin{problem}
Let \(x, y\) be nonnegative real numbers such that \(x + y = 5\) and \(xy = 7\). Find
\(\dfrac{x}{y-1} +\dfrac{y}{p-1}\)
\end{problem}

\begin{problem}
Let \(a, b\) are real numbers such that
\[
    \dfrac{1}{a(b+1)} + \dfrac{1}{b(a+1)} = \dfrac{1}{(a+1)(b+1)}.
\]
\end{problem}

\begin{problem}
Find the sum of roots to the equation \(x^{2020} = 2020x^{2019} + 1\)
\end{problem}

\begin{problem}
Find the product of the roots of the equation \(x^3 = 9\pi x + x^2 + 1. \)
\end{problem}

\begin{problem}
What is the average value of the three roots of the equation \(x^3 -12x^2 - 4x + 48 = 0\)?
\end{problem}

\end{document}