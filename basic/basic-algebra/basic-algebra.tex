\documentclass[a4paper]{article}
\usepackage[english]{babel}
\usepackage{graphicx}
\usepackage{multicol}
\usepackage{amsmath}
\usepackage{hyperref}
\usepackage{amsthm}
\usepackage{geometry}
\geometry{a4paper} 
\usepackage{fancyhdr}
\usepackage{xcolor}
\usepackage{amssymb}
\usepackage{multicol}
\theoremstyle{definition}
\newtheorem{exmp}{Example}[section]
\newtheorem{theorem}{Theorem}

\begin{document}
\author{Fractals}
\title{\textbf{}}
\maketitle
\tableofcontents
\noindent
\section{Basic Algebra}
\subsection{Simon's Factroring Trick}
\textbf{Simon’s Favorite Factoring Trick} (SFFT) is best explained with an example:
\begin{exmp}
    Find all positive integers \(x, y\) that satisfy
    \[
        xy- 2x - 4y = 0.
    \]
    \(Sloution\) Let us factor the first two terms:
    \[
        x(y-2) - 4y = 0.
    \]
    We want to find some way we can turn the \(y\) into a \(y - 2\). Let’s see what
    happens if we do that:
    \[
        x(y-2) - 4(y-2+2) = 0.
    \]
    \[
        x(y-2) - 4(y-2)  - 8= 0.
    \]
    \[
        x(y-2) - 4(y-2) = 8.
    \]
    Now, we can factor:
    \[
        (x-4)(y-2) = 8.
    \]
    Because \(x, y\) are positive integers,
    we know that \(x - 4 \text{ and } y - 2\) are simply the positive factors of 8
    \[
        x-4 = 1, y-2=8,
    \]
    \[
        x-4 = 2, y-2=4,
    \]
    \[
        x-4 = 4, y-2=2,
    \]
    \[
        x-4 = 8, y-2=1,
    \]
    Solving we get \framebox[1.04\width]{\((x,y)\in  \{(5,10), (6,6), (8,4), (12,3)\} \).}
    \\

    \noindent
    Now for the formal statement:
\end{exmp}
\begin{theorem}[\textbf{SFFT}]
    For all real numbers (although commonly used only for integers) \(x, y, a, b, \)
    \[
        xy + xa + yb + ab = (x + a)(y + bk).
    \]
    Two special common cases are: \(xy + x + y + 1 = (x+1)(y+1)\) and \(xy - x - y + 1 = (x-1)(y-1)\).
\end{theorem}
\end{document}